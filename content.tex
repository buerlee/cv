%----------------------------------------------------------------------------------------
%      HEADER
%----------------------------------------------------------------------------------------
\author{Martin Schrimpf}
\title{\theauthor\space}


%----------------------------------------------------------------------------------------
%	PERSONAL
%----------------------------------------------------------------------------------------
\setAddress{%
	33 3rd Ave\\% \#204
	Boston\\%
	MA 02129}
%	Oberhausmehring 16\\%
%	84405 Dorfen\\%
%	Germany}
\setEmail{%
%	martin.schrimpf@outlook.com
	msch@mit.edu}
\setMobile{%
	+1 617-586-6748}
\setWeb{%
	www.mschrimpf.com}


%----------------------------------------------------------------------------------------
%	EDUCATION
%----------------------------------------------------------------------------------------
\newcommand{\education}{%
\educationblock%
{% 1: Time frame start
Since 09/2017
}{% 2: Time frame end
}{% 3: Program
PhD student
}{% 4: University
MIT Brain and Cognitive Sciences (BCS)
}{% 5: Description
Research at the bridge of Machine Learning and Neuroscience.
Current rotation with James DiCarlo and Joshua Tenenbaum on visual decomposition in deep learning models as well as the brain's ventral stream.
Future projects will be in the direction of learning.
}{% 6: Technologies
}{% 7: Grade
}

\educationblock%
{% 1: Time frame start
10/2014
}{% 2: Time frame end
05/2017
}{% 3: Program
Elite Master's Program Software Engineering
}{% 4: University
\\ 
TU \& LMU Munich \& University of Augsburg
}{% 5: Description
Formal Methods, Distributed Systems, Project Management, Databases, Human Computer Interaction. 
Extra courses in Machine Learning.
\\
\subprogram{Master's Thesis} at \organization{Harvard University} on the role of brain-inspired recurrent neural algorithms for advanced object recognition.
}{% 6: Technologies
Java, C++, SQL, QVTO, Maude, CTL, LTL
}{% 7: Grade
$4.0$ w/ honors
}

\educationblock%
{% 1: Time frame start
10/2011
}{% 2: Time frame end
07/2014
}{% 3: Program
Bachelor's Program Information Systems
}{% 4: University
TU Munich
}{% 5: Description
Combination of economic fundamentals and computer science with a focus on Information Systems.
\\
\subprogram{Bachelor's Thesis} at the \organization{University of Sydney}: Investigation of hardware transactional memory and its effectiveness as a synchronization technique for databases, graded $A+$.
\\
\subprogram{Study Abroad} at the \organization{Auckland University of Technology}: Courses in Artificial Intelligence and Management.
Project on a novel landmark-based approach to perceptual mapping (SLAM) at CAIR.
}{% 6: Technologies
Java, UML, SQL, C++, C, Assembly, ARIA
}{% 7: Grade
$3.8$
}

\educationblock%
{% 1: Time frame start
2003
}{% 2: Time frame end
2011
}{% 3: Program
Abitur
}{% 4: University
Gymnasium Dorfen
}{% 5: Description
Focus on Mathematics, Computer Science, Economics, English.
\subprogram{USA-exchange} with the \organization{C.D. Hylton High school} in Virginia
}{% 6: Technologies
}{% 7: Grade
$3.7$
}
}

\newcommand{\varEducationInfo}{All grades converted into the American 4-point system}


%----------------------------------------------------------------------------------------
%	EXPERIENCE
%----------------------------------------------------------------------------------------
\newcommand{\professional}{%
\block%
{% 1: Time frame start
05/2017
}{% 2: Time frame end
08/2017
}{% 3: Position
Deep Learning Intern
}{% 4: Employer
Salesforce Einstein AI (former MetaMind)
}{% 5: Description
Research in the emerging field of automated architecture search.
We took a flexible approach to define a recurrent architecture and found several architectures that do not follow human intuition yet outperform the state-of-the-art model.
}{% 6: Technologies
PyTorch, AWS
}

\block%
{% 1: Time frame start
04/2016
}{% 2: Time frame end
11/2016
}{% 3: Position
Research Assistant
}{% 4: Employer
Kreiman Lab, Harvard Medical School
}{% 5: Description
Research at the bridge of machine learning, neuroscience and cognitive science with a focus on the role of recurrent connections.
We improved object recognition performance on occluded objects from 45\% with Alexnet to 74\% with our models and offered a first possible application of recurrency in the human brain.
We are also investigating the robustness of deep convolutional networks and the role of context for object recognition
}{% 6: Technologies
Matlab, Python, keras, Theano, Linux, LSF, ECoG
}

\block%
{% 1: Time frame start
12/2015
}{% 2: Time frame end
04/2016
}{% 3: Position
Research Assistant
}{% 4: Employer
Oracle Labs
}{% 5: Description
Enabled research teams to flexibly utilize the Oracle RDBMS on the internal cluster by developing an on-demand database module
}{% 6: Technologies
Linux, LSF, Virtual Machines
}

\block%
{% 1: Time frame start
07/2015
}{% 2: Time frame end
10/2015
}{% 3: Position
Software Engineering Intern
}{% 4: Employer
Siemens AG
}{% 5: Description
Architectural concept and development of a behavior-driven testing framework that can run a test specification written in natural language and that is now used in three major business areas
}{% 6: Technologies
Python
}

\block%
{% 1: Time frame start
07/2012
}{% 2: Time frame end
12/2015
}{% 3: Position
Freelancer
}{% 4: Employer
Martin Schrimpf Software Solutions
}{% 5: Description
Software Development and Services - projects include:
\begin{description}[noitemsep,topsep=0pt,font=\normalfont\itshape\space]
\item[Greimel IT-Systemhaus GmbH] Led the development of a Document Management System including optical character recognition (OCR), a financial accounting interface and a dynamic workflow and process management system which made the client company effectively paper-free
%\item[R-Backup Datensicherung GmbH] Developed a multilingual website's front- and backend to administrate partners and customers and to issue invoices
\item[Promonde JLT] Implemented an advertisement website for Arabic countries with over 10,000 users per day
\end{description}
}{% 6: Technologies
Java, JavaScript, PHP
}
}


%----------------------------------------------------------------------------------------
%	PUBLICATIONS
%----------------------------------------------------------------------------------------
\newcommand{\publications}{%
\block%
{% 1: Time frame start
2017
}{% 2: Time frame end
}{% 3: Activity
}{% 4: Organisation
}{% 5: Description
\vspace{-8pt} % this is horrible but the bibentry command seems to generate some vertical distance and I don't know how to remove it
\fullcite{schrimpf2017}
}{% 6: Technologies
}

\block%
{% 1: Time frame start
2017
}{% 2: Time frame end
}{% 3: Activity
}{% 4: Organisation
}{% 5: Description
\vspace{-8pt}
\fullcite{cheney2017robustness}
}{% 6: Technologies
}

\block%
{% 1: Time frame start
2017
}{% 2: Time frame end
}{% 3: Activity
}{% 4: Organisation
}{% 5: Description
\vspace{-8pt}
\fullcite{recurrency_occlusion}
}{% 6: Technologies
}

\block%
{% 1: Time frame start
2014
}{% 2: Time frame end
}{% 3: Activity
}{% 4: Organisation
}{% 5: Description
\vspace{-8pt}
\fullcite{bachelors_thesis}
}{% 6: Technologies
}
}


%----------------------------------------------------------------------------------------
%	PRESENTATIONS
%----------------------------------------------------------------------------------------
\newcommand{\presentations}{%
\block%
{% 1: Time frame start
12/2016
}{% 2: Time frame end
}{% 3: Activity
}{% 4: Organisation
Brains \& Bits, NIPS Workshops
}{% 5: Description
Recurrent computations for pattern completion
}{% 6: Technologies
}

\block%
{% 1: Time frame start
10/2016
}{% 2: Time frame end
}{% 3: Activity
}{% 4: Organisation
Systems Club, Harvard Medical School
}{% 5: Description
Recurrent computations for pattern completion
}{% 6: Technologies
}
}


%----------------------------------------------------------------------------------------
%	AWARDS
%----------------------------------------------------------------------------------------
\newcommand{\awards}{%
\block%
{% 1: Time frame start
2017
}{% 2: Time frame end
}{% 3: Activity
Social Impact Award (Integreat)
}{% 4: Organisation
TUM School of Management
}{% 5: Description
}{% 6: Technologies
}

\block%
{% 1: Time frame start
2016
}{% 2: Time frame end
}{% 3: Activity
FITweltweit
}{% 4: Organisation
DAAD German Academic Exchange Service
}{% 5: Description
}{% 6: Technologies
}

\block%
{% 1: Time frame start
2016
}{% 2: Time frame end
}{% 3: Activity
Teilstipendium
}{% 4: Organisation
University Augsburg
}{% 5: Description
}{% 6: Technologies
}

\block%
{% 1: Time frame start
2016
}{% 2: Time frame end
}{% 3: Activity
Integrationspreis (Integreat)
}{% 4: Organisation
Government of Swabia
}{% 5: Description
}{% 6: Technologies
}

\block%
{% 1: Time frame start
2016
}{% 2: Time frame end
}{% 3: Activity
Winner Social Society (Integreat)
}{% 4: Organisation
Idea- and Startup-competition Generation-D
}{% 5: Description
}{% 6: Technologies
}

\block%
{% 1: Time frame start
2015
}{% 2: Time frame end
}{% 3: Activity
Deutschlandstipendium
}{% 4: Organisation
Federal Ministry for Education and Research, Roland und Ute Lacher Fonds
}{% 5: Description
}{% 6: Technologies
}

\block%
{% 1: Time frame start
2014
}{% 2: Time frame end
}{% 3: Activity
Ministeriumsstipendium
}{% 4: Organisation
Bavarian State Ministry for Education, Science and the Arts
}{% 5: Description
}{% 6: Technologies
}

\block%
{% 1: Time frame start
2013
}{% 2: Time frame end
2016
}{% 3: Activity
e-fellows.net scholarship
}{% 4: Organisation
}{% 5: Description
}{% 6: Technologies
}
}


%----------------------------------------------------------------------------------------
%	EXTRACURRICULAR ACTIVITIES
%----------------------------------------------------------------------------------------
\newcommand{\extracurricular}{%
\block%
{% 1: Time frame start
02/2016
}{% 2: Time frame end
}{% 3: Activity
Artificial Intelligence Workshop
}{% 4: Organisation
}{% 5: Description
Organized a two-day workshop on Neural Networks, Machine Learning and Organic Computing. The speakers were Prof. Günther Palm, PD Rolf Würtz and Dr. Joschka Bödecker
}{% 6: Technologies
}

\block%
{% 1: Time frame start
Since 08/2015
}{% 2: Time frame end
}{% 3: Activity
Co-Founder and Technical Lead
}{% 4: Organisation
Integreat
}{% 5: Description
Platform to deliver information from local authorities and helper organizations to refugees in over 80 German cities.
Implementation of the administration backend and a cross-platform app, coordination of the development community
}{% 6: Technologies
Xamarin (C\#), Android (Java), WordPress (PHP)
}

\block%
{% 1: Time frame start
2015
}{% 2: Time frame end
2016
}{% 3: Activity
MINGA Mentor for International Students
}{% 4: Organisation
TU Munich
}{% 5: Description
}{% 6: Technologies
}

\block%
{% 1: Time frame start
2013
}{% 2: Time frame end
2016
}{% 3: Activity
Rotaract Club München Residenz
}{% 4: Organisation
}{% 5: Description
Youth club of Rotary: community, helping and learning. Social initiatives, e.g. with our orphanage sponsorship
}{% 6: Technologies
}
}


%----------------------------------------------------------------------------------------
%      LANGUAGES
%----------------------------------------------------------------------------------------
\newcommand{\languages}{
\languageproficiency{German}{Native proficiency}
\languageproficiency{English}{Full professional proficiency}
\languageproficiency{Japanese}{Elementary proficiency}
\languageproficiency{French}{Elementary proficiency}
}


%----------------------------------------------------------------------------------------
%      INTERESTS
%----------------------------------------------------------------------------------------
\newcommand{\interests}{
\interest{Travelling}{Insights into various cultures in places such as Africa, Australia and India}
\interest{Martial Arts}{Sporty balance, perfection of techniques and meditation with Judo and Shaolin}
\interest{Brain-inspired Computing}{Getting behind the concepts of cognition and intelligence on the basis of biological findings, side projects in e.g. deep reinforcement learning and home automation}
}


%----------------------------------------------------------------------------------------
%	ADVISED STUDENTS
%----------------------------------------------------------------------------------------
\newcommand{\advisedstudents}{%
\block%
{% 1: Time frame start
Fall 2016
}{% 2: Time frame end
}{% 3: who
Jacklyn Sarette
}{% 4: Organisation
Emmanuel College
}{% 5: Description
Behavioral experiments on visual context
}{% 6: Technologies
}

\block%
{% 1: Time frame start
Fall 2016
}{% 2: Time frame end
}{% 3: who
Doré de Morsier
}{% 4: Organisation
ETH Zurich
}{% 5: Description
Behavioral experiments on the recognition of novel objects
}{% 6: Technologies
}

\block%
{% 1: Time frame start
Summer 2016
}{% 2: Time frame end
}{% 3: who
Wendy Fernandez
}{% 4: Organisation
City University of New York
}{% 5: Description
Behavioral experiments and data analysis on the identification of occluded objects (MIT Summer Research Program)
}{% 6: Technologies
}
}


%----------------------------------------------------------------------------------------
%	REFERENCES
%----------------------------------------------------------------------------------------

\newcommand{\references}{%
\block%
{% 1: Time frame start
}{% 2: Time frame end
}{% 3: who
Prof. Gabriel Kreiman, PhD
}{% 4: Organisation
Children's Hospital Boston, Harvard Medical School
}{% 5: Description
}{% 6: Technologies
}

\iffalse
\block%
{% 1: Time frame start
}{% 2: Time frame end
}{% 3: who
Prof. Uwe Röhm, PhD
}{% 4: Organisation
School of IT, University of Sydney
}{% 5: Description
}{% 6: Technologies
}
\fi

\block%
{% 1: Time frame start
}{% 2: Time frame end
}{% 3: who
Prof. Dr. Helmut Krcmar
}{% 4: Organisation
Computer science in economics, Technical University Munich
}{% 5: Description
}{% 6: Technologies
}

\block%
{% 1: Time frame start
}{% 2: Time frame end
}{% 3: who
Prof. Dr. Alexander Knapp
}{% 4: Organisation
Software and Systems Engineering, Augsburg University
}{% 5: Description
}{% 6: Technologies
}
}
