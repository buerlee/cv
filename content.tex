%----------------------------------------------------------------------------------------
%	HEADER
%----------------------------------------------------------------------------------------
\author{Martin Schrimpf}
\title{\theauthor\space}

%----------------------------------------------------------------------------------------
%	PERSONAL
%----------------------------------------------------------------------------------------
\setAddress{%
	Oberhausmehring 16\\%
	84405 Dorfen\\%
	Germany}
\setEmail{%
	martin.schrimpf@tum.de}
\setMobile{%
	+49 174 9946449}
\setGithub{%
	https://github.com/mschrimpf}


%----------------------------------------------------------------------------------------
%	EDUCATION
%----------------------------------------------------------------------------------------
\newcommand{\education}{%
\educationblock%
{% 1: Time frame start
10/2014
}[% 2: Time frame end
10/2016
]{% 3: Program
Elite Master's Program Software Engineering
}[% 4: University
\\ 
TU Munich, LMU Munich and University of Augsburg
][% 5: Description
Elite Graduate Program for a small group of students with a focus on the areas Formal Methods, Project Management, Distributed Systems, Databases, Human Computer Interaction and soft-skills.
\\
\subprogram{Master's Thesis} at the \organization{University of Sydney}: Investigation of hardware transactional memory and its effectiveness as a synchronization technique for databases, graded $1.0$.
][% 6: Technologies
Java, C++, SQL, QVTO, Maude, CTL, LTL
][% 7: Grade
$\varnothing 1.1$
]

\educationblock%
{% 1: Time frame start
10/2011
}[% 2: Time frame end
07/2014]
{% 3: Program
Bachelor's Program Information Systems
}[% 4: University
TU Munich, $\varnothing 1.4$
][% 5: Description
Combination of economic fundamentals and computer science with a focus on Information Systems.
\\
\subprogram{Bachelor's Thesis} at the \organization{University of Sydney}: Investigation of hardware transactional memory and its effectiveness as a synchronization technique for databases, graded $1.0$.
\\
\subprogram{Study Abroad} at the \organization{Auckland University of Technology}: Courses in Artificial Intelligence and Management.
Project on a novel landmark-based approach to perceptual mapping (SLAM) at CAIR
][% 6: Technologies
Java, UML, SQL, C++, C, Assembly, ARIA
]

\educationblock%
{% 1: Time frame start
2003
}[% 2: Time frame end
2011]
{% 3: Program
Abitur
}[% 4: University
Gymnasium Dorfen
][% 5: Description
Focus on Mathematics, Computer Science, Economics, English.
\subprogram{USA-exchange} with the \organization{C.D. Hylton High school} in Virginia
][% 6: Technologies
]
}

\newcommand{\varEducationInfo}{All grades in the German scale (1.0 excellent - 5.0 fail)}


%----------------------------------------------------------------------------------------
%	EXPERIENCE
%----------------------------------------------------------------------------------------
\newcommand{\professional}{%
\block%
{% 1: Time frame start
04/2016
}[% 2: Time frame end
11/2016
]{% 3: Position
Research Assistant
}[% 4: Employer
Kreiman Lab, Harvard University
][% 5: Description
Research at the bridge of machine learning, neuroscience and psychology with a focus on the role of recurrent connections.
We improved object detection performance on occluded objects from 45\% with Alexnet to 74\% with our models and offered a first possible application of recurrency in the human brain [paper submitted].
A follow-up project focuses on the role of scene context for object detection
][% 6: Technologies
Matlab, Python, keras, Theano, Linux, ECoG
]

\block%
{% 1: Time frame start
12/2015
}[% 2: Time frame end
04/2016
]{% 3: Position
Research Assistant
}[% 4: Employer
Oracle Labs
][% 5: Description
Developed an on-demand database module for the computing cluster that is now used by internal research teams
][% 6: Technologies
Linux, LSF, Virtual Machines
]

\block%
{% 1: Time frame start
07/2015
}[% 2: Time frame end
10/2015
]{% 3: Position
Software Engineering Intern
}[% 4: Employer
Siemens AG
][% 5: Description
Architectural concept and implementation of a behavior-driven testing framework that can run a test specification written in natural language and that is now used in three major business areas
][% 6: Technologies
Python
]

\block%
{% 1: Time frame start
07/2012
}[% 2: Time frame end
12/2015
]{% 3: Position
Freelancer
}[% 4: Employer
Martin Schrimpf Software Solutions
][% 5: Description
Software Development and Services - projects include:
\begin{description}[noitemsep,topsep=0pt,font=\normalfont\itshape\space]
\item[Greimel IT-Systemhaus GmbH] Led the development of a Document Management System including optical character recognition (OCR), a financial accounting interface and a dynamic workflow management system
\item[R-Backup Datensicherung GmbH] Developed a multilingual website's front- and backend to administrate partners and customers and to issue invoices
\item[Deutsche Flugsicherung GmbH] Developed a tool for exercise preparation, containing best-route-finding in thousands of waypoints and pre-simulation of the entire exercise.
The project unfortunately never went live due to contract constraints
\item[Promonde JLT] Developed an advertisement website for Arabic countries with over 10k users per day
\end{description}
][% 6: Technologies
Java, JavaScript, PHP
]
}


%----------------------------------------------------------------------------------------
%	PUBLICATIONS
%----------------------------------------------------------------------------------------
\newcommand{\publications}{%
\block%
{% 1: Time frame start
2016
}[% 2: Time frame end
]{% 3: Activity
}[% 4: Organisation
][% 5: Description
\vspace{-8pt} % this is horrible but the bibentry command seems to generate some vertical distance and I don't know how to remove it
\bibentry{recurrency_occlusion}
][% 6: Technologies
]

\block%
{% 1: Time frame start
2016
}[% 2: Time frame end
]{% 3: Activity
}[% 4: Organisation
][% 5: Description
\vspace{-8pt}
\bibentry{should_i_use_tensorflow}
][% 6: Technologies
]

\block%
{% 1: Time frame start
2014
}[% 2: Time frame end
]{% 3: Activity
}[% 4: Organisation
][% 5: Description
\vspace{-8pt}
\bibentry{bachelors_thesis}
][% 6: Technologies
]
}


%----------------------------------------------------------------------------------------
%	PRESENTATIONS
%----------------------------------------------------------------------------------------
\newcommand{\presentations}{%
\block%
{% 1: Time frame start
2016
}[% 2: Time frame end
]{% 3: Activity
Presentation
}[% 4: Organisation
NIPS Brains \& Bits
][% 5: Description
Recurrent computations for pattern completion
][% 6: Technologies
]

\block%
{% 1: Time frame start
2016
}[% 2: Time frame end
]{% 3: Activity
Presentation
}[% 4: Organisation
Harvard Systems Club
][% 5: Description
Recurrent computations for pattern completion
][% 6: Technologies
]
}


%----------------------------------------------------------------------------------------
%	AWARDS
%----------------------------------------------------------------------------------------
\newcommand{\awards}{%
\block%
{% 1: Time frame start
2016
}[% 2: Time frame end
]{% 3: Activity
FITweltweit
}[% 4: Organisation
DAAD German Academic Exchange Service
][% 5: Description
][% 6: Technologies
]

\block%
{% 1: Time frame start
2016
}[% 2: Time frame end
]{% 3: Activity
Teilstipendium
}[% 4: Organisation
University Augsburg
][% 5: Description
][% 6: Technologies
]

\block%
{% 1: Time frame start
2016
}[% 2: Time frame end
]{% 3: Activity
Winner Social Society
}[% 4: Organisation
Idea- and Startup-competition Generation-D
][% 5: Description
][% 6: Technologies
]

\block%
{% 1: Time frame start
2015
}[% 2: Time frame end
]{% 3: Activity
Deutschlandstipendium
}[% 4: Organisation
Federal Ministry for Education and Research, Roland und Ute Lacher Fonds
][% 5: Description
][% 6: Technologies
]

\block%
{% 1: Time frame start
2014
}[% 2: Time frame end
]{% 3: Activity
Ministeriumsstipendium
}[% 4: Organisation
Bavarian State Ministry for Education, Science and the Arts
][% 5: Description
][% 6: Technologies
]

\block%
{% 1: Time frame start
201\{3,4,5,6\}
}[% 2: Time frame end
]{% 3: Activity
e-fellows.net scholarship
}[% 4: Organisation
][% 5: Description
][% 6: Technologies
]
}


%----------------------------------------------------------------------------------------
%	EXTRACURRICULAR ACTIVITIES
%----------------------------------------------------------------------------------------
\newcommand{\extracurricular}{%
\block%
{% 1: Time frame start
02/2016
}[% 2: Time frame end
]{% 3: Activity
Artificial Intelligence Workshop
}[% 4: Organisation
][% 5: Description
Organized a two-day workshop on Neural Networks, Machine Learning and Organic Computing. The speakers were Prof. Günther Palm, PD Rolf Würtz and Dr. Joschka Bödecker
][% 6: Technologies
]

\block%
{% 1: Time frame start
Since 08/2015
}[% 2: Time frame end
]{% 3: Activity
Co-Founder and Technical Lead
}[% 4: Organisation
Integreat
][% 5: Description
Platform to deliver information from local authorities and helper organizations to refugees in over 80 German cities.
Implementation of the administration backend and a cross-platform app, later coordination of the development community
][% 6: Technologies
Xamarin (C\#), Android (Java), WordPress (PHP)
]

\block%
{% 1: Time frame start
201\{5,6\}
}[% 2: Time frame end
]{% 3: Activity
MINGA Mentor for International Students
}[% 4: Organisation
TU Munich
][% 5: Description
][% 6: Technologies
]

\block%
{% 1: Time frame start
Since 10/2013
}[% 2: Time frame end
]{% 3: Activity
Rotaract Club München Residenz
}[% 4: Organisation
][% 5: Description
Youth club of Rotary, based on the community, helping and learning. Social initiatives, e.g. with our orphanage sponsorship
][% 6: Technologies
]
}


%----------------------------------------------------------------------------------------
%	ADVISED STUDENTS
%----------------------------------------------------------------------------------------
\newcommand{\advisedstudents}{%
\block%
{% 1: Time frame start
2016
}[% 2: Time frame end
]{% 3: who
Jacklyn Sarette
}[% 4: Organisation
Emmanuel College
][% 5: Description
Behavioral experiments on visual context
][% 6: Technologies
]

\block%
{% 1: Time frame start
2016
}[% 2: Time frame end
]{% 3: who
Doré de Morsier
}[% 4: Organisation
ETH Zurich
][% 5: Description
Behavioral experiments on the recognition of novel objects
][% 6: Technologies
]

\block%
{% 1: Time frame start
2016
}[% 2: Time frame end
]{% 3: who
Wendy Fernandez
}[% 4: Organisation
City University of New York
][% 5: Description
Behavioral experiments and data analysis on the identification of occluded objects
][% 6: Technologies
]
}


%----------------------------------------------------------------------------------------
%	REFERENCES
%----------------------------------------------------------------------------------------
\newcommand{\references}{%
\block%
{% 1: Time frame start
}[% 2: Time frame end
]{% 3: who
Prof. Gabriel Kreiman, PhD
}[% 4: Organisation
Children’s Hospital Boston, Harvard Medical School
][% 5: Description
][% 6: Technologies
]

\block%
{% 1: Time frame start
}[% 2: Time frame end
]{% 3: who
Prof. Uwe Röhm, PhD
}[% 4: Organisation
School of IT, University of Sydney
][% 5: Description
][% 6: Technologies
]

\block%
{% 1: Time frame start
}[% 2: Time frame end
]{% 3: who
Prof. Dr. Helmut Krcmar
}[% 4: Organisation
Computer science in economics, Technical University Munich
][% 5: Description
][% 6: Technologies
]
}
